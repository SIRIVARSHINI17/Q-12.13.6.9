\let\negmedspace\undefined
\let\negthickspace\undefined
\documentclass[journal,12pt,twocolumn]{IEEEtran}
\usepackage{cite}
\usepackage{amsmath,amssymb,amsfonts,amsthm}
\usepackage{algorithmic}
\usepackage{graphicx}
\usepackage{textcomp}
\usepackage{xcolor}
\usepackage{txfonts}
\usepackage{listings}
\usepackage{enumitem}
\usepackage{mathtools}
\usepackage{gensymb}
\usepackage[breaklinks=true]{hyperref}
\usepackage{tkz-euclide} % loads  TikZ and tkz-base
\usepackage{listings}

%\bibliographystyle{ieeetr}
\begin{document}
%
\providecommand{\pr}[1]{\ensuremath{\Pr\left(#1\right)}}
\providecommand{\prt}[2]{\ensuremath{p_{#1}^{\left(#2\right)} }}        % own macro for this question
\providecommand{\qfunc}[1]{\ensuremath{Q\left(#1\right)}}
\providecommand{\sbrak}[1]{\ensuremath{{}\left[#1\right]}}
\providecommand{\lsbrak}[1]{\ensuremath{{}\left[#1\right.}}
\providecommand{\rsbrak}[1]{\ensuremath{{}\left.#1\right]}}
\providecommand{\brak}[1]{\ensuremath{\left(#1\right)}}
\providecommand{\lbrak}[1]{\ensuremath{\left(#1\right.}}
\providecommand{\rbrak}[1]{\ensuremath{\left.#1\right)}}
\providecommand{\cbrak}[1]{\ensuremath{\left\{#1\right\}}}
\providecommand{\lcbrak}[1]{\ensuremath{\left\{#1\right.}}
\providecommand{\rcbrak}[1]{\ensuremath{\left.#1\right\}}}
\newcommand{\sgn}{\mathop{\mathrm{sgn}}}
\providecommand{\abs}[1]{\left\vert#1\right\vert}
\providecommand{\res}[1]{\Res\displaylimits_{#1}} 
\providecommand{\norm}[1]{\left\lVert#1\right\rVert}
%\providecommand{\norm}[1]{\lVert#1\rVert}
\providecommand{\mtx}[1]{\mathbf{#1}}
\providecommand{\mean}[1]{E\left[ #1 \right]}
\providecommand{\cond}[2]{#1\middle|#2}
\providecommand{\fourier}{\overset{\mathcal{F}}{ \rightleftharpoons}}
\newenvironment{amatrix}[1]{%
  \left(\begin{array}{@{}*{#1}{c}|c@{}}
}{%
  \end{array}\right)
}
%\providecommand{\hilbert}{\overset{\mathcal{H}}{ \rightleftharpoons}}
%\providecommand{\system}{\overset{\mathcal{H}}{ \longleftrightarrow}}
	%\newcommand{\solution}[2]{\textbf{Solution:}{#1}}
\newcommand{\solution}{\noindent \textbf{Solution: }}
\newcommand{\cosec}{\,\text{cosec}\,}
\providecommand{\dec}[2]{\ensuremath{\overset{#1}{\underset{#2}{\gtrless}}}}
\newcommand{\myvec}[1]{\ensuremath{\begin{pmatrix}#1\end{pmatrix}}}
\newcommand{\mydet}[1]{\ensuremath{\begin{vmatrix}#1\end{vmatrix}}}
\newcommand{\myaugvec}[2]{\ensuremath{\begin{amatrix}{#1}#2\end{amatrix}}}
\providecommand{\rank}{\text{rank}}
\providecommand{\pr}[1]{\ensuremath{\Pr\left(#1\right)}}
\providecommand{\qfunc}[1]{\ensuremath{Q\left(#1\right)}}
	\newcommand*{\permcomb}[4][0mu]{{{}^{#3}\mkern#1#2_{#4}}}
\newcommand*{\perm}[1][-3mu]{\permcomb[#1]{P}}
\newcommand*{\comb}[1][-1mu]{\permcomb[#1]{C}}
\providecommand{\qfunc}[1]{\ensuremath{Q\left(#1\right)}}
\providecommand{\gauss}[2]{\mathcal{N}\ensuremath{\left(#1,#2\right)}}
\providecommand{\diff}[2]{\ensuremath{\frac{d{#1}}{d{#2}}}}
\providecommand{\myceil}[1]{\left \lceil #1 \right \rceil }
\newcommand\figref{Fig.~\ref}
\newcommand\tabref{Table~\ref}
\newcommand{\sinc}{\,\text{sinc}\,}
\newcommand{\rect}{\,\text{rect}\,}
%%
%	%\newcommand{\solution}[2]{\textbf{Solution:}{#1}}
%\newcommand{\solution}{\noindent \textbf{Solution: }}
%\newcommand{\cosec}{\,\text{cosec}\,}
%\numberwithin{equation}{section}
%\numberwithin{equation}{subsection}
%\numberwithin{problem}{section}
%\numberwithin{definition}{section}
%\makeatletter
%\@addtoreset{figure}{problem}
%\makeatother

%\let\StandardTheFigure\thefigure
\let\vec\mathbf

\bibliographystyle{IEEEtran}


\vspace{3cm}

\title{

\textbf{QUESTION : 12.13.6.9} 

}
\author{ ROLL NO:EE22BTECH11027\\
         NAME: KATARI SIRI VARSHINI
	}
	
	

\maketitle

\newpage

%\tableofcontents

\bigskip

\renewcommand{\thefigure}{\theenumi}
\renewcommand{\thetable}{\theenumi}
12.13.6.9.An experiment succeeds twice as often as it fails. Find the probability that in the next six trials, there will be atleast 4 successes.
\\ \solution:
Let p be the probability for the experiment to succeed and q for the failure.\\
Here, it is given that probability of success is twice that of the failure, so 
\begin{align}
\begin{split}
p&=2q\\
q&= \frac{1}{3}\\
p&= \frac{2}{3}
\end{split}
\end{align}
Now, let's consider a single trial as a bernuolli random variable $X_{i}=1$ represents success and $X_{i}=0$ represents failure. 
\begin{center}
\begin{tabular}{|c|c|c|}
\hline
Parameter & Value & Description  \\
\hline
$X_i$ \cbrak{0,1} & 1 & success\\
$X_i$ \cbrak{0,1} & 0 & failure\\
\hline
\end{tabular}
\end{center}
Therefore we have,
\begin{align}
P(X_i=1)= p= \frac{2}{3}\\
P(X_i=0)= q= \frac{1}{3}
\end{align}
Since we have n=6 trials, the random variable X representing the number of successes in 6 trials follows a binomial distribution. The cumulative distribution function (CDF) of X is given by
\begin{equation}
\label{eq:bd}
F_X(k)= P_X(X \le k)= \Sigma_{k=0}^n\comb{n}{k}q^{n-k}p^{k}
\end{equation}
Here, $$n=6, p= \frac{2}{3}, q= \frac{1}{3}, k= 0,1,2...6$$\\
We need to find the probability for the experiment to succeed to atleast 4 times i.e. $P_X(X \ge 4)$.
Using equation \ref{eq:bd} we get,
\begin{multline}
\begin{split}
P_X(X \ge 4)&= 1-P_X(X \le 3)\\
&=1-F_X(3)\\
&=1-( \comb{6}{0}\left(\frac{1}{3}\right)^{6}\left(\frac{2}{3}\right)^{0}+ \comb{6}{1}\left(\frac{1}{3}\right)^{5}\left(\frac{2}{3}\right)^{1}\\&+ \comb{6}{2}\left(\frac{1}{3}\right)^{4}\left(\frac{2}{3}\right)^{2}+ \comb{6}{3}\left(\frac{1}{3}\right)^{3}\left(\frac{2}{3}\right)^{3})\\
&= 1-\frac{233}{3^{6}} \approxeq 0.680
\end{split}
\end{multline}
Therefore the probability that in the next six trials, there will be atleast 4 successes is 0.680.
\end{document}
